\chapter{Using elPaSo Core}

As already mentioned in the introduction, elPaSo Core is can be used in two ways according to your purpose:
\begin{enumerate}
    \item \textbf{eCore Executable}: The executable can be used to solve vibroacoustic problems. Guideline to compile an elpaso executable is outlined in Section \ref{sec:eCoreAsSolver}.
    \item \textbf{eCore Library}: By generating eCore library, you can link eCore as a third party library in your project and use the FEM functionalies from eCore.
\end{enumerate}

\section{elPaSo Core as a vibroacoustic solver} \label{sec:eCoreAsSolver}

elPaSo can be with GNU compilers and Intel compilers (recommended). This section mainly focus on elPaSo preparation using a basic GNU compiler.

\subsection{Executable types}
\begin{enumerate}
    \item \textbf{elpaso}: Executable compiled with real data-types and hence computations without complex domain (mainly used for time-domain computations).
    \item \textbf{elpasoC}: Executable compiled with complex data type.
    \item \textbf{elpasoT}: Executable running all the unit-tests for eCore using GoogleTest.
\end{enumerate}

\subsection{Building executable}
\begin{enumerate}
    \item Download and run the docker image containing the basic environment required for building elPaSo. This contains the necessary third part libraries and compiler.
    \begin{itemize}
        \item \texttt{sudo docker pull <link to the image will follow soon after intial release>}
        \item \texttt{sudo docker run -it <image name>}
        \item Now you are inside the container with all necessary elPaSo dependencies.
    \end{itemize}
    \item Checkout the current version of the elPaSo Core project
    \begin{itemize}
        \item \texttt{git clone https://git.rz.tu-bs.de/akustik/elPaSo-Core}
        \item \texttt{cd elPaSo-Core}
    \end{itemize}
    \item Configure a cmake configuration file for your system
    \begin{itemize}
        \item \texttt{make cmake-gen}
        \item This creates a config.cmake file with your hostname in the "cmake" folder. The file contains the default configuration. You need not change anything here unless required.
    \end{itemize}
    \item Build elPaSo Core executable
    \begin{enumerate}
        \item With GNU compiler
        \begin{itemize}
            \item \texttt{mkdir project}
            \item \texttt{make elpaso-gnu}
            \item \texttt{make elpaso-build}
            \item When sucessfull, executables (elpaso, elpasoC and elpasoT) are generated in ./bin folder. 
        \end{itemize}
        \item With Intel compiler
        \begin{itemize}
            \item Install intel-parallel-studio and configure the ./cmake/\$HOSTNAME.config.cmake with the respective directory where the installation is done with INTEL\_DIR and the respective version number to INTELMPI\_VERSION.
            \item \texttt{mkdir project}
            \item \texttt{make elpaso-intel}
            \item \texttt{make elpaso-build}
            \item Executables (elpaso, elpasoC and elpasoT) are generated in ./bin folder. 
        \end{itemize}
    \end{enumerate}
\end{enumerate}

\subsection{Running executable}
elPaSo can be started by direct use of the compiled files elpaso and elpasoC. The input file is
needed in the folder where the calculation is started. A simple serial run with standard settings
can be made by the command: \\

{\hspace{2em} \tt elpasoC -c -inp myInputFile.hdf5}\\

A check of the input file is started by the command: \\

{\hspace{2em} \tt elpasoC -check -inp myInputFile.hdf5}\\

To run problems with the executable elpaso having real-only petsc datatype, follow the same commands with the respective executable name: \\

{\hspace{2em} \tt elpaso -c -inp myInputFile.hdf5}\\

Unit-tests can be performed by running the test executable: \\

{\hspace{2em} \tt elpasoT}\\

\subsection{Running executable in parallel using MPI}
In order to execute elpaso in parallel may some additional MPI commands have to be used. These commands
are depending on the computer used, e.g.,\\

{\hspace{2em} \tt mpirun -np 4 elpasoC -c -inp myInputFile.hdf5}\\

in order to run the program using 4 processes. \texttt{mpirun} is the instance supplied from an MPI vendor. For GNU compiler build, we use OpenMPI and for INTEL compiler build, we use Intel MPI.

\section{elPaSo Core as a FEM library} \label{sec:eCoreAsLibrary}
\subsection{Building library}

\begin{enumerate}
    \item Download and run the docker image containing the basic environment required for building elPaSo. This contains the necessary third part libraries and compiler.
    \begin{itemize}
        \item \texttt{sudo docker pull <link to the image will follow soon after intial release>}
        \item \texttt{sudo docker run -it <image name>}
        \item Now you are inside the container with all necessary elPaSo dependencies.
    \end{itemize}
    \item Checkout the current version of the elPaSo Core project
    \begin{itemize}
        \item \texttt{git clone https://git.rz.tu-bs.de/akustik/elPaSo-Core}
        \item \texttt{cd elPaSo-Core}
    \end{itemize}
    \item Configure a cmake configuration file for your system
    \begin{itemize}
        \item \texttt{make cmake-gen}
        \item This creates a config.cmake file with your hostname in the "cmake" folder. The file contains the default configuration.
        \item To generate the eCore library, active the library generation option in ./cmake/\$HOSTNAME.config.cmake by modifying: \texttt{OPTION(GEN\_DLIB			"elPaSo DYNAMIC LIB"	ON)}
    \end{itemize}
    \item Build elPaSo Core library
    \begin{enumerate}
        \item With GNU compiler
        \begin{itemize}
            \item \texttt{mkdir project}
            \item \texttt{make elpaso-gnu}
            \item \texttt{make elpasolib-build}
            \item When sucessfull, elpasoCore-gnu folder contains the necessary entities for linking eCore.
        \end{itemize}
        \item With Intel compiler
        \begin{itemize}
            \item Install intel-parallel-studio and configure the ./cmake/\$HOSTNAME.config.cmake with the respective directory where the installation is done with INTEL\_DIR and the respective version number to INTELMPI\_VERSION.
            \item \texttt{mkdir project}
            \item \texttt{make elpaso-intel}
            \item \texttt{make elpasolib-build}
            \item When sucessfull, elpasoCore-intel folder contains the necessary entities for linking eCore.
        \end{itemize}
    \end{enumerate}
\end{enumerate}

\subsection{Linking library}
Follow the steps below, to link eCore library with CMake. In your FindELPASOCORE.cmake:
\begin{enumerate}
    \item Include the path \texttt{./elpasoCore-<COMPILER>/include} to your cmake project for link to header files
    \item Link the eCore dynamic library:
    \begin{itemize}
        \item For real-valued build, link \texttt{./elpasoCore-<COMPILER>/lib/libelpasoCore-intel-cxx-o.so}
        \item For complex-valued build, link \texttt{./elpasoCore-<COMPILER>/lib/libelpasoCore-intel-cxx-complex.so}
    \end{itemize}
\end{enumerate}